\documentclass[10pt, oneside]{article} 

\usepackage{amsmath, amsthm, amssymb,amsfonts,appendix}
\usepackage{bbm, bm,booktabs}
\usepackage{calrsfs,color,cite,caption,ctex}
\usepackage{esint,enumitem}
\usepackage{fancyhdr,fontspec,float}
\usepackage{graphicx,graphics,geometry}
\usepackage{indentfirst}
\usepackage{listings}
\definecolor{dkgreen}{rgb}{0,0.6,0}
\definecolor{gray}{rgb}{0.5,0.5,0.5}
\definecolor{mauve}{rgb}{0.58,0,0.82}

\lstset{frame=tb,
	language=Python,
	aboveskip=3mm,
	belowskip=3mm,
	showstringspaces=false,
	columns=flexible,
	basicstyle={\small\ttfamily},
	numbers=none,
	numberstyle=\tiny\color{gray},
	keywordstyle=\color{blue},
	commentstyle=\color{dkgreen},
	stringstyle=\color{mauve},
	breaklines=true,
	breakatwhitespace=true,
	tabsize=3
}
\usepackage[labelfont=bf]{caption}
\usepackage{mathptmx}
\usepackage{stmaryrd,siunitx,subfigure,setspace}
\usepackage[stable]{footmisc}
\usepackage{tikz,textcomp}
\usetikzlibrary{fit,positioning,arrows,automata}
\usepackage{verbatim}
\usepackage{wasysym}
\usepackage{wrapfig}



\geometry{tmargin=.75in, bmargin=.75in, lmargin=.75in, rmargin = .75in}  



\newtheorem{thm}{Theorem}
\newtheorem{defn}{Definition}
\newtheorem{conv}{Convention}
\newtheorem{rem}{Remark}
\newtheorem{lem}{Lemma}
\newtheorem{cor}{Corollary}


\title{
Notes
}
\author{171240510 Yuheng Ma\\[0.3cm]{.}}
\date{Summer 2020}

\begin{document}

\maketitle

\vspace{.25in}

\section*{Notation}
\begin{itemize}
\item N number of time series
\item T length of time series
\item $y_t$ N$\times$1 vector of observation at $\mathbf{t}$
\item $f_t$ q$\times$1 factor at $\mathbf{t}$
\item B $N\times q$ loading matrix
\end{itemize}

我们考虑加权主成分方法(Weighted PCA)。对p维观察数据$\mathbf{y}_{t}=\left(y_{1 t}, \ldots, y_{p t}\right)^{\prime} \in \mathbb{R}^{p}$,有$$\mathbf{y}_{t}=\mathbf{B} \mathbf{f}_{t}+\mathbf{u}_{t}, \quad t=1, \ldots, T$$其中$\mathbf{f}_{t} \in \mathbb{R}^{K}$是K维公共因子,B为因子载荷矩阵。根据Bai and Liao 2013,$$\widehat{\mathbf{F}}=\left(\widehat{\mathbf{f}}_{1}, \ldots, \widehat{\mathbf{f}}_{T}\right)^{\prime}$$
$$\widehat{\mathbf{B}}=T^{-1} \mathbf{Y} \widehat{\mathbf{F}}$$
即为F与B的相合估计。

\section*{Method}
PCA:
\begin{equation}
\operatorname{argmin}_{\mathbf{f}_{\mathbf{f}} \in \mathbb{R}^{K}} \sum_{t=1}^{T}\left(\mathbf{y}_{t}-\mathbf{B f}_{t}\right)^{\prime}\left(\mathbf{y}_{t}-\mathbf{B} \mathbf{f}_{t}\right)
\end{equation}
\par
WPCA:
\begin{equation}
\operatorname{argmin}_{\mathbf{f}_{\mathbf{f}} \in \mathbb{R}^{K}} \sum_{t=1}^{T}\left(\mathbf{y}_{t}-\mathbf{B f}_{t}\right)^{\prime} \Sigma_{u}^{-1}\left(\mathbf{y}_{t}-\mathbf{B} \mathbf{f}_{t}\right)
\end{equation}

\par 
Karlman Filter

%%尝试的方法包括:

%对Doz(2011)等

\section*{Experiment}

Cholesky decomposition is not positive definite. 
\begin{figure}[H]

  \includegraphics[width=\linewidth]{error.png}

  \label{fig: error}
    \end{figure}

\section*{Reference}

\begin{itemize}
\item Quefeng Li, Guang Cheng, Jianqing Fan \& Yuyan Wang (2018) Embracing the Blessing of Dimensionality in Factor Models, Journal of the American Statistical Association, 113:521, 380-389, DOI: 10.1080/01621459.2016.1256815


\item Bai, J., and Liao, Y. (2013), “Statistical Inferences Using Large Estimated Covariances for Panel Data and Factor Models,” arXiv:1307.2662. [380,381,382,383]
\end{itemize}

\end{document}
